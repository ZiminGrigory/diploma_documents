%% Простая презентация с примером включения программного кода и
%% пошаговых спецэффектов
\documentclass{beamer}
\usepackage{fontspec}
\usepackage{xunicode}
\usepackage{xltxtra}
\usepackage{xecyr}
\usepackage{hyperref}
\setmainfont[Mapping=tex-text]{DejaVu Serif}
\setsansfont[Mapping=tex-text]{DejaVu Sans}
\setmonofont[Mapping=tex-text]{DejaVu Sans Mono}
\usepackage{polyglossia}
\setdefaultlanguage{russian}
\setbeamertemplate{footline}[frame number]
\setbeamertemplate{navigation symbols}{}
\usepackage{graphicx}
\usepackage{listings}
\usepackage{caption}

\begin{document}
\title{Создание языка программирования роботов в терминах потоков данных с применением DSM-подхода}
\thispagestyle{empty}
\author{Григорий Александрович Зимин\\{\footnotesize{Научный руководитель: д. ф.-м. н., профессор А.Н. Терехов \\Рецензент: асс. каф. КСиПТ СПбПУ М.А. Беляев 
}}}
\institute{Санкт-Петербургский государственный университет\\Математико-механический факультет\\Кафедра Системного программирования}
\date{Санкт-Петербург, \today}
\frame{\titlepage}

\begin{frame}\frametitle{Модели исполнения}
\centering 
\begin{columns}[T]
     \begin{column}[T]{5cm} 
     		\begin{figure}[p]
    			\centering
          		\includegraphics[scale = 0.3]{cf.png}
    			\caption*{Поток управления}
    			\label{fig:cf}
			\end{figure}
     \end{column}
     \begin{column}[T]{5cm} 
     		\begin{figure}[p]
  				\begin{center}
          		\includegraphics[scale = 0.2]{df.png}
          		\end{center}
          		\caption*{Поток данных}
    			\label{fig:df}
			\end{figure}
     \end{column}
\end{columns}
\end{frame}

\begin{frame}\frametitle{Управление роботом}
	\centering 	\includegraphics[scale = 0.65]{robotControl.png}
\end{frame}



\begin{frame}\frametitle{Инструменты для программирования роботов}
\centering 
\begin{itemize}
	\item Индустриальные
	\begin{itemize}
    	\item Microsoft Robotics Developer Studio (Microsoft)
    	\item LabVIEW (National Instruments)
    	\item Simulink (MathWorks)
	\end{itemize}
	\item Учебные
	\begin{itemize}
    	\item NXT-G (LEGO)
    	\item ROBOLAB (LEGO)
    	\item TRIK Studio (СПбГУ)
	\end{itemize}
	\item Академические
	\begin{itemize}
    	\item Builder + MCAGUI + MCABrowser (M. Proetzsch, T. Luksch, K. Berns) 2007
    	\item POPed (J. Simpson, C.L. Jacobsen) 2008
    	\item RuRu (J.P. Diprose, B.A. MacDonald, J. Hosking) 2011
	\end{itemize}
\end{itemize}
\end{frame}

\begin{frame}\frametitle{Постановка задачи}

\begin{itemize}
    \item Создать новый визуальный язык программирования в терминах потоков данных для программирования роботов
    \begin{itemize}
        \item TRIK
        \item NXT
        \item EV3
    \end{itemize}
    \item Интерпретировать программы, написанные на новом языке
    \begin{itemize}
        \item Двумерная модель
        \item Реальный робот
    \end{itemize}
    \item Апробировать на типовых системах управления
\end{itemize}
\end{frame}

	
%\begin{frame}\frametitle{Модели декомпозиции управления роботом}
%	\begin{columns}[T]
%     \begin{column}[T]{6cm} 
%     		\begin{figure}[p]
%    			\centering
%          		\includegraphics[scale = 0.3]{commonModel.png}
%    			\caption*{Модель до Р. Брукса}
%    			\label{fig:cf}
%			\end{figure}
%     \end{column}
%     \begin{column}[T]{6cm} 
%     		\begin{figure}[p]
%  				\begin{center}
%          		\includegraphics[scale = 0.3]{brooksModel.png}
%          		\end{center}
%          		\caption*{Модель Р. Брукса}
%    			\label{fig:df}
%			\end{figure}
%     \end{column}
%\end{columns}
%\end{frame}

\begin{frame}\frametitle{Архитектура системы управления роботом}
\centering 
     		\begin{figure}[p]
				\includegraphics[scale = 0.25]{brooksArchitecture.png}
          		\caption*{Архитектура Р. Брукса\footnote{Изображение взято из статьи автора.}}
			\end{figure}
\end{frame}


\begin{frame}\frametitle{Язык}
 		\centering 
  \begin{figure}
    \includegraphics[scale = 0.6]{langElments}
    \caption*{Элементы языка: связь, блок.}
  \end{figure}

\end{frame}

\begin{frame}\frametitle{Блоки \textit{действий} с роботом}
\centering 
	
\begin{figure}
    \includegraphics[scale = 0.5]{pic/msg}
    \caption*{\textit{Моторы, Сенсор, Геймпад}}
\end{figure}

\end{frame}


\begin{frame}\frametitle{Блоки \textit{манипулирования потоками}}
\centering 
	
\begin{figure}
    \includegraphics[scale = 0.4]{pic/szuz}
    \caption*{\textit{Подавление и замещение, Запаковка, Распаковка}}
\end{figure}

\end{frame}



\begin{frame}\frametitle{\textit{Управляющие} блоки}
\centering 
	
\begin{figure}
    \includegraphics[scale = 0.3]{pic/fso}
    \caption*{\textit{Распараллеливание, Пользовательский блок, Выходной порт}}
\end{figure}

\end{frame}

\begin{frame}\frametitle{\textit{Управляющие} блоки}
\centering 
	
\begin{figure}
\centering 
    \includegraphics[scale = 0.26]{pic/xeff}
    \caption*{\textit{Глобальная переменная, Завершение исполнения} \\ \textit{Фильтр, Текстовое программирование}}
\end{figure}

\end{frame}



\begin{frame}\frametitle{Реализация}
 \begin{columns}[T]
     \begin{column}[T]{5cm} 
     	\begin{itemize}
			\item Редактор диаграмм
    		\item Интерпретатор диаграмм в терминах потоков данных
		\end{itemize}
     \end{column}
     \begin{column}[T]{5cm} 
          \includegraphics[height=5cm]{Common.png}
     \end{column}
\end{columns}
\end{frame}


\begin{frame}\frametitle{Реализация}\framesubtitle{Архитектура интерпретатора}
\centering
          \includegraphics[scale=0.4]{pic/InterpreterArch.png}
\end{frame}


\begin{frame}\frametitle{Реализация}\framesubtitle{Интерпретация}
\begin{itemize}
			\item Подготовка
    		\item Исполнение
		\end{itemize}

\centering
          \includegraphics[scale=0.4]{pic/Interaction.png}

\end{frame}



%\begin{frame}\frametitle{Апробация: ПД-регулятор для движения вдоль стены}\framesubtitle{Система управления}
%\centering 
%\begin{columns}[T]
%     \begin{column}[T]{5cm} 
%     		\begin{figure}[p]
%    			\centering
%          		\includegraphics[scale = 0.34]{alongCF.png}
%    			\caption*{Поток управления}
%    			\label{fig:cf}
%			\end{figure}
%     \end{column}
%     \begin{column}[T]{5cm} 
%     		\begin{figure}[p]
%  				\begin{center}
%          		\includegraphics[scale = 0.2]{alongBoxCode.png}
%          		\end{center}
%          		\caption*{Поток данных}
%    			\label{fig:df}
%			\end{figure}
%     \end{column}
%\end{columns}
%\end{frame}


\begin{frame}\frametitle{Апробация: симуляция типовой задачи управления}
		\centering 	\includegraphics[scale = 0.22]{working.png}
\end{frame}


%\begin{frame}\frametitle{Апробация: ПД-регулятор для движения вдоль стены}\framesubtitle{Интерпретация на роботе}
%	\centering
%	\includegraphics[scale = 0.2]{alongBoxReal.png}
%\end{frame}


\begin{frame}\frametitle{Апробация: двухуровневая система управления роботом}\framesubtitle{Операторский контроль}
		\centering 	\includegraphics[scale = 0.4]{pult.png}
\end{frame}


\begin{frame}\frametitle{Апробация: двухуровневая система управления роботом}\framesubtitle{Избегание столкновений}
		\centering \includegraphics[scale = 0.4]{collision.png}
\end{frame}


\begin{frame}\frametitle{Апробация: двухуровневая система управления роботом}\framesubtitle{Управление на основе архитектуры категорий}
		\centering \includegraphics[scale = 0.4]{programScreen.png}
\end{frame}

\begin{frame}\frametitle{Апробация: двухуровневая система управления роботом}\framesubtitle{Интерпретация на роботе}
	\centering
	\includegraphics[scale = 0.16]{simulationReal.png}
\end{frame}


\begin{frame}\frametitle{Результаты}
\begin{itemize}
	\item Создан новый визуальный язык программирования в терминах потоков данных для программирования роботов и редактор для него
    \item Для интерпретации программ на новом языке на двумерной модели и на реальном роботе создан интерпретатор потоковых языков
    \item Произведена апробация на типовых задачах управления роботом
    \item Написаны две публикации на тематические конференции (SEIM-2016, SYRCoSE-2016) 
\end{itemize}
\end{frame}
\end{document}