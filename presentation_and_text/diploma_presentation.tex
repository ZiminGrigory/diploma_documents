%% Простая презентация с примером включения программного кода и
%% пошаговых спецэффектов
\documentclass{beamer}
\usepackage{fontspec}
\usepackage{xunicode}
\usepackage{xltxtra}
\usepackage{xecyr}
\usepackage{hyperref}
\setmainfont[Mapping=tex-text]{DejaVu Serif}
\setsansfont[Mapping=tex-text]{DejaVu Sans}
\setmonofont[Mapping=tex-text]{DejaVu Sans Mono}
\usepackage{polyglossia}
\setdefaultlanguage{russian}
\setbeamertemplate{footline}[page number]{}
\setbeamertemplate{navigation symbols}{}
\usepackage{graphicx}
\usepackage{listings}
\usepackage{caption}

\begin{document}
\title{Создание языка программирования роботов в терминах потоков данных с применением DSM-подхода}
\author{Григорий Александрович Зимин\\{\footnotesize{Научный руководитель: д. ф.-м. н., профессор А.Н. Терехов \\Рецензент:?
}}}
\institute{Санкт-Петербургский государственный университет\\Математико-механический факультет\\Кафедра Системного программирования}
\date{Санкт-Петербург, \today}
\frame{\titlepage}

\begin{frame}\frametitle{Модели исполнения}
\begin{columns}[T]
     \begin{column}[T]{5cm} 
     		\begin{figure}[p]
    			\centering
          		\includegraphics[scale = 0.3]{cf.png}
    			\caption*{Поток управления}
    			\label{fig:cf}
			\end{figure}
     \end{column}
     \begin{column}[T]{5cm} 
     		\begin{figure}[p]
  				\begin{center}
          		\includegraphics[scale = 0.2]{df.png}
          		\end{center}
          		\caption*{Поток данных}
    			\label{fig:df}
			\end{figure}
     \end{column}
\end{columns}
\end{frame}


\begin{frame}\frametitle{Управление роботом}
	\includegraphics[scale = 0.65]{robotControl.png}
\end{frame}
	
\begin{frame}\frametitle{Модели декомпозиции управления роботом}
	\begin{columns}[T]
     \begin{column}[T]{6cm} 
     		\begin{figure}[p]
    			\centering
          		\includegraphics[scale = 0.3]{commonModel.png}
    			\caption*{Модель до Р. Брукса}
    			\label{fig:cf}
			\end{figure}
     \end{column}
     \begin{column}[T]{6cm} 
     		\begin{figure}[p]
  				\begin{center}
          		\includegraphics[scale = 0.3]{brooksModel.png}
          		\end{center}
          		\caption*{Модель Р. Брукса}
    			\label{fig:df}
			\end{figure}
     \end{column}
\end{columns}
\end{frame}

\begin{frame}\frametitle{Архитектура системы управления роботом}
     		\begin{figure}[p]
				\includegraphics[scale = 0.25]{brooksArchitecture.png}
          		\caption*{Архитектура Р. Брукса\footnote{Изображение взято из статьи автора.}}
			\end{figure}
\end{frame}

\begin{frame}\frametitle{Исследования в области}
\begin{itemize}
    \item Proetzsch Martin, Luksch Tobias, Berns Karsten. The behaviour-based control architecture iB2C for complex robotic systems. 2007  
    \item Jonathan Simpson, Christian L. Jacobsen. Visual Process-oriented Programming for Robotics. 2008
    \item Diprose James P, MacDonald Bruce A, Hosking John G. Ruru: A spatial and interactive visual programming language for novice robot programming. 2011
\end{itemize}
\end{frame}


\begin{frame}\frametitle{Среды программирования роботов}
\begin{itemize}
	\item Основаны на модели потока данных
	\begin{itemize}
    	\item Microsoft Robotics Developer Studio (Microsoft)
    	\item LabVIEW (National Instruments)
    	\item Simulink (MAthWorks)
	\end{itemize}
	\item Основаны на модели потока управления
	\begin{itemize}
    	\item NXT-G (LEGO)
    	\item ROBOLAB (LEGO)
    	\item TRIK Studio (СПбГУ)
	\end{itemize}
\end{itemize}
\end{frame}


\begin{frame}\frametitle{Постановка задачи}
\begin{itemize}
    \item Создать новый визуальный язык программирования в терминах потоков данных для программирования роботов
    \begin{itemize}
        \item TRIK
        \item NXT
        \item EV3
    \end{itemize}
    \item Интерпретировать программы, написанные на новом языке
    \begin{itemize}
        \item Двумерная модель
        \item Реальный робот
    \end{itemize}
    \item Апробировать на типовых системах управления
\end{itemize}
\end{frame}


\begin{frame}\frametitle{Язык}
\only<1>{  
  \begin{figure}
    \includegraphics[scale = 0.6]{langElments}
    \caption*{Элементы языка: связь, блок.}
  \end{figure}
}

\only<2>{
	{\bf Блоки \textit{действий} с роботом}
 \begin{columns}[T] 
	
	
	\begin{column}[T]{3cm} 
  		\begin{figure}
    		\includegraphics[scale = 0.6]{motors}
   			\caption*{\scriptsize\textit{Моторы}} 
 		\end{figure}
     \end{column}
     
     \begin{column}[T]{3cm} 
  		\begin{figure}
    		\includegraphics[scale = 0.6]{sensor}
   			\caption*{\scriptsize\textit{Сенсор}} 
 		\end{figure}
     \end{column}  
        
     \begin{column}[T]{3cm} 
  		\begin{figure}
    		\includegraphics[scale = 0.6]{gamepad}
   			\caption*{\scriptsize\textit{Геймпад}} 
 		\end{figure}
     \end{column}
  \end{columns}
}

\only<3>{
	{\bf Блоки \textit{манипулирования потоками}}
 \begin{columns}[T] 
	
	\begin{column}[T]{2cm} 
  		\begin{figure}
    		\includegraphics[scale = 0.6]{zip}
   			\caption*{\scriptsize\textit{Запаковка}} 
 		\end{figure}
     \end{column}
     
     \begin{column}[T]{2cm} 
  		\begin{figure}
    		\includegraphics[scale = 0.6]{unzip}
   			\caption*{\scriptsize\textit{Распаковка}} 
 		\end{figure}
     \end{column}  
        
     \begin{column}[T]{3cm} 
  		\begin{figure}
    		\includegraphics[scale = 0.6]{supress}
   			\caption*{\scriptsize\textit{Подавление и замещение}} 
 		\end{figure}
     \end{column}
  \end{columns}
}

\only<4>{
 {\bf \textit{Управляющие} блоки}
 \begin{columns}[T] 
	
	 \begin{column}[T]{3cm} 
  		\begin{figure}
    		\includegraphics[scale = 0.6]{fork}
   			\caption*{\scriptsize\textit{Распараллеливание}} 
 		\end{figure}
     \end{column}  
     
	\begin{column}[T]{3cm} 
  		\begin{figure}
    		\includegraphics[scale = 0.6]{subprog}
   			\caption*{\scriptsize\textit{Пользовательский блок}} 
 		\end{figure}
     \end{column}
        
     \begin{column}[T]{3cm} 
  		\begin{figure}
    		\includegraphics[scale = 0.6]{outport}
   			\caption*{\scriptsize\textit{Выходной порт}} 
 		\end{figure}
     \end{column}
  \end{columns}
}

\only<5>{
	{\bf \textit{Управляющие} блоки}  
  \begin{columns}[T] 
	\begin{column}[T]{2cm} 
  		\begin{figure}
    		\includegraphics[scale = 0.4]{setx}
   			\caption*{\scriptsize\textit{Глобальная переменная}} 
 		\end{figure}
     \end{column}
     
     \begin{column}[T]{2cm} 
  		\begin{figure}
    		\includegraphics[scale = 0.4]{exit}
   			\caption*{\scriptsize\textit{Завершение исполнения}} 
 		\end{figure}
     \end{column}  
  \end{columns}
  
 \begin{columns}[T] 	
	\begin{column}[T]{2cm} 
  		\begin{figure}
    		\includegraphics[scale = 0.4]{delayFilter}
   			\caption*{\scriptsize\textit{Фильтр}} 
 		\end{figure}
     \end{column}
     
     \begin{column}[T]{2cm} 
  		\begin{figure}
    		\includegraphics[scale = 0.35]{func}
   			\caption*{\scriptsize\textit{Текстовое программирование}} 
 		\end{figure}
     \end{column}  
  \end{columns}

}
\end{frame}


\begin{frame}\frametitle{Реализация}
 \begin{columns}[T]
     \begin{column}[T]{5cm} 
     	\begin{itemize}
			\item Редактор диаграмм
    		\item Интерпретатор диаграмм в терминах потоков данных
		\end{itemize}
     \end{column}
     \begin{column}[T]{5cm} 
          \includegraphics[height=5cm]{Common.png}
     \end{column}
\end{columns}
\end{frame}


\begin{frame}\frametitle{Апробация: ПД-регулятор для движения вдоль стены}\framesubtitle{Система управления}
\begin{columns}[T]
     \begin{column}[T]{5cm} 
     		\begin{figure}[p]
    			\centering
          		\includegraphics[scale = 0.34]{alongCF.png}
    			\caption*{Поток управления}
    			\label{fig:cf}
			\end{figure}
     \end{column}
     \begin{column}[T]{5cm} 
     		\begin{figure}[p]
  				\begin{center}
          		\includegraphics[scale = 0.2]{alongBoxCode.png}
          		\end{center}
          		\caption*{Поток данных}
    			\label{fig:df}
			\end{figure}
     \end{column}
\end{columns}
\end{frame}


\begin{frame}\frametitle{Апробация: ПД-регулятор для движения вдоль стены}\framesubtitle{Интерпретация на модели}
	\includegraphics[scale = 0.22]{working.png}
\end{frame}


\begin{frame}\frametitle{Апробация: ПД-регулятор для движения вдоль стены}\framesubtitle{Интерпретация на роботе}

\end{frame}


\begin{frame}\frametitle{Апробация: двухуровневая система управления роботом}\framesubtitle{Операторский контроль}
	\includegraphics[scale = 0.4]{pult.png}
\end{frame}


\begin{frame}\frametitle{Апробация: двухуровневая система управления роботом}\framesubtitle{Избегание столкновений}
\includegraphics[scale = 0.4]{collision.png}
\end{frame}


\begin{frame}\frametitle{Апробация: двухуровневая система управления роботом}\framesubtitle{Управление на основе архитектуры категорий}
	\includegraphics[scale = 0.4]{programScreen.png}
\end{frame}


\begin{frame}\frametitle{Результаты}
\begin{itemize}
	\item Создан новый визуальный язык программирования в терминах потоков данных для программирования роботов и редактор для него
    \item Для интерпретации программ на новом языке на двумерной модели и на реальном роботе создан интерпретатор потоковых языков
    \item Произведена апробация на типовых задачах управления роботом
    \item Написаны две публикации на тематические конференции (SEIM-2016, SYRCoSE-2016) 
\end{itemize}
\end{frame}
\end{document}