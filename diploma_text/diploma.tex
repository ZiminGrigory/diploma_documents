% Тут используется класс, установленный на сервере Papeeria. На случай, если
% текст понадобится редактировать где-то в другом месте, рядом лежит файл matmex-diploma-custom.cls
% который в момент своего создания был идентичен классу, установленному на сервере.
% Для того, чтобы им воспользоваться, замените matmex-diploma на matmex-diploma-custom
% Если вы работаете исключительно в Papeeria то мы настоятельно рекомендуем пользоваться
% классом matmex-diploma, поскольку он будет автоматически обновляться по мере внесения корректив
%

% По умолчанию используется шрифт 14 размера. Если нужен 12-й шрифт, уберите опцию [14pt]
%\documentclass[14pt]{matmex-diploma}
\documentclass[14pt]{matmex-diploma-custom}

\begin{document}
% Год, город, название университета и факультета предопределены,
% но можно и поменять.
% Если англоязычная титульная страница не нужна, то ее можно просто удалить.
\filltitle{ru}{
    chair              = {Кафедра Системного программирования},
    title              = {Создание языка программирования роботов в терминах потоков данных с применением DSM-подхода},
    % Здесь указывается тип работы. Возможные значения:
    %   coursework - Курсовая работа
    %   diploma - Диплом специалиста
    %   master - Диплом магистра
    %   bachelor - Диплом бакалавра
    type               = {bachelor},
    position           = {студента},
    group              = 444,
    author             = {Зимин Григорий Александрович},
    supervisorPosition = {д.\,ф.-м.\,н., профессор},
    supervisor         = {Терехов А.\,Н.},
    reviewerPosition   = {},
    reviewer           = {Беляев М.\,А.},
    %chairHeadPosition  = {д.\,ф.-м.\,н., профессор},
    %chairHead          = {Хунта К.\,Х.},
%   university         = {Санкт-Петербургский Государственный Университет},
%   faculty            = {Математико-механический факультет},
%   city               = {Санкт-Петербург},
%   year               = {2013}
}
\filltitle{en}{
    chair              = {Chair of Software Engineering},
    title              = {Development of dataflow robotics programming language using DSM-approach},
    type               = {bachelor},
    author             = {Grigorii Zimin},
    supervisorPosition = {Professor},
    supervisor         = {Terekhov A.\,N.},
    reviewerPosition   = {},
    reviewer           = {Belyaev M.\,A.},
%    chairHeadPosition  = {},
%    chairHead          = {},
}

\maketitle
\tableofcontents

% У введения нет номера главы
\section*{Введение}

В настоящее время интерес к конструированию роботов и управлению ими растет. В этой области проводится множество исследований: на крупнейших конференциях, посвященных робототехнике, таких как IROS~\cite{IROS}, ICRA~\cite{ICRA}, исследовательские группы со всего мира обсуждают новые подходы к решению различных проблем в робототехнике. Также на протяжении последних трех десятилетий, проводится исследование возможностей применения визуальных языков программирования (visual programming languages), результаты исследований публикуются на крупнейших конференциях, к примеру, на симпозиуме VL/HCC~\cite{IROS}. Применение визуальных языков программирования в робототехнике --- область в которой также существует множество исследований~\cite{banyasad2000visual, simpson2006mobile, simpson2008visual, posso2011process, diprose2011ruru}. Визуальные языки программирования в робототехнике позволяют сокращать время создания систем управления роботами, а также нагляднее их отображать. Это в частности используется для обучения школьников или новичков программированию роботов: есть несколько сред для обучения программированию роботов, например, ROBOLAB~\cite{robolab}, NXT-G~\cite{nxt-g}, TRIK Studio~\cite{trik}, которые позволяют программировать поведение робота с помощью модельно-ориентированного подхода, где для описания программы используется набор моделей, чаще всего визуальных, имитирующих высокоуровневые паттерны поведения робота. 


Система управления роботом может быть рассмотрена как взаимодействие трех составляющих: датчики и сенсоры, логика системы управления, приводы. Датчики и сенсоры генерируют данные, логика системы управления собирает значения, обрабатывает их и генерирует импульсы для приводов. Отсюда следует, что по своей природе программы управления роботами реактивны: они обрабатывают сигналы, непрерывно приходящие с датчиков и сенсоров, и генерируют управляющую информацию для приводов. По сути они решают задачу трансформации данных. Для программирования таких задач хорошо подходят потоковые или реактивные языки программирования, они же --- языки программирования потоков данных (data flow languages). Данные языки в свою очередь также активно эволюционировали от текстовых языков к визуальным языкам потоков данных, которые сейчас широко распространены~\cite{johnston2004advances}. При программировании потоков данных наглядность визуальных языков превосходит текстовые, так как они явно отображают потоки данных на диаграмме. В индустрии программирования роботов существует несколько широко распространенных, крупных и довольно-таки сложных сред программирования, к примеру, Simulink~\cite{Simulink}, LabVIEW~\cite{LabVIEW}, они предоставляют пользователю большой и даже порой громоздкий набор средств и библиотек для программирования различных роботов (подробный обзор языков программирования роботов приведен в главе~\ref{sec:overview}).




Мы хотим создать язык, отвечающий реактивной природе роботов и всем аспектам, описанным выше. Понятно, что мы хотим создать узконаправленный инструмент для решения конкретной задачи, а не специфицировать решения на каком-либо языке программирования общего назначения. Создать новый предметно-ориентированный язык на основе предметно-ориентированного моделирования. Для более быстрого создания таких языков используют DSM-платформы~\cite{kelly2008domain}. Примерами таких платформ являются: MetaEdit+~\cite{metaEdit}, Modeling SDK for Microsoft Visual Studio 2015~\cite{MVSTools}. В данной работе будет использоваться DSM-платформа QReal~\cite{терехов2009архитектура}, которая разрабатывается на кафедре Системного программирования Санкт-Петербургского Государственного Университета, предоставляющая средства поддержки визуальных предметно-ориентированных языков(Domain-Specific Modelling, DSM-подход).


\section*{Постановка задачи}
    % Задача курсовой --- спецификация и прототипирование такого языка, у тебя описана задача ВКР, но всем пофиг.
Задачей данной курсовой работы стало создание языка управления роботами в терминах потока управления данными, используя подход предметно-ориентированного моделирования на базе DSM-платформы QReal: его подробная спецификация, метамодель --- описание всех связей и сущностей языка, а также правила построения визуальных моделей по ним, генерация по модели кода в любой общепринятый язык программирования, поддерживающий или эмулирующий реактивное программирование~\cite{wikiReactiveProg}. Язык должен быть расширением языка TRIK, следовательно все простейшие задачи должны также выполняться и на нем, более того, это не должно стать сложнее, а даже проще в случаях, где управление данными сокращает программу и делает её более понятной. Во-вторых, основываясь на том, что сейчас очень популярна операционная система ROS~\cite{ros}, которая предоставляет разработчикам библиотеки и инструменты для создания приложений робототехники, аппаратную абстракцию, драйверы устройств, визуализаторы и многое другое. Необходимо создать инструмент для генерации по спецификациям ROS приложений для робототехники в нашем новом языке. В третьих, требуется провести аппробацию языка и генератора кода, создав, к примеру, систему контроля, отвечающую за управление сегвеем.

\section{Обзор}
\label{sec:overview}
 Визуальные языки в терминах потоков данных не новы. Можно рассмотреть, к примеру:
 \begin{itemize}    
  \item Систему разработки виртуальных приборов LabView~\cite{travis2007labview}
  \item Визульный событийно ориентированный язык программирования~\cite{simpson2008visual}
  \item Диссертация "Система для визуального программирования автономных роботов"~\cite{banyasad2000visual}
 \end{itemize}

\bibliographystyle{utf8gost705u}
\bibliography{diploma.bib}
\end{document}
